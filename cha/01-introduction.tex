% \section{Context}
Since its first mention in 2006 \cite{Adelmann2006ToolkitFB} the \ac{IoT}
is a growing industry sector and active research field. A 2018 study conducted
by Dachyar et al., shows that \ac{IoT} has been continuously growing, is expected
to grow further, and will rise in demand for research and commercial application
because of its function to connect a variety of devices. The primary industries
where \ac{IoT} devices and applications are used today are manufacturing,
agriculture, public service, electronics, health, energy, and mining.
While most research is focused on topics such as security, protocols, energy and
power consumption \cite{dachyar2019knowledge}.

\ac{IoT} devices that are deployed in the mining and energy industry are often
installed in locations with limited, unreliable, or expensive network
connectivity, such as off-shore oil rigs or underground mining sites.
Some other industries have similar challenging installation sites, such as
satellites, cargo ships, and remote weather stations.
The limited bandwidth available in these remote locations poses a significant
obstacle for transmitting large amounts of data reliably and efficiently,
requiring innovative solutions for data optimization.

To address these challenges, there are already several approaches to compress
the transmitted data without a loss of information, which have shown to reduce
the data transmitted, but also require more computing power on the \ac{IoT}
devices \cite{9243457}. An additional technique to reduce the amount of data
is to use a more efficient serialization format, such as Google's \textit{ProtoBuf}
which has shown to reduce the size of data transmitted for a particular application
by 83\% and 80\% in comparison to \textit{JSON} and \textit{BSON} respectively \cite{7765670}.
Lastly, when switching to an edge computing approach, the data can be preprocessed
and aggregated before sending it to the cloud, which reduces the amount of data
transmitted.


\section{Problem}
As applications often gain functionality and complexity with each release, the
size of software may also increase. \ac{IoT} devices operating in environments
with limited bandwidth will face increasing challenges. One case where this is
particularly evident is the \textit{Linux} kernel used by many \ac{IoT} and embedded
devices, which expands in size with every new release \cite{linux-kernel-report}.

However, product managers and developers are advised to update their devices with
the latest software since
research has shown that updating security issues in \ac{IoT} systems as quickly
and as frequently as possible significantly enhances the security of these systems.
This is especially important for \ac{IIoT} devices, which have an extended attack
surface and where an outage can impose physical and financial damage \cite{s20247160}.
For \ac{IoT} devices with limited connectivity, crucial security updates can be
significantly delayed or even impossible to apply, as the update size may exceed
the available bandwidth or disk storage. But not just the security of \ac{IoT}
devices is improved by frequent updating. Anand et al. showed in 2018 that the
performance and reliability of \ac{IoT} applications can be significantly improved
by updating frequently \cite{Anand2018}.

When updating the \ac{OS} for an \ac{IoT} device, it is very common to use an
A/B partition scheme, which will keep the device operational while downloading
and installing the latest version to the inactive partition and then switch with
a reboot once the update is complete. However, this approach is impractical for
devices which have limited storage, as the inactive partition will consume
half of the available storage. Further and most importantly, when updating
it is required to download an entire bootable \ac{OS} image which can be
hundreds of \ac{MB} in size. This is not feasible for devices with limited
internet connection.

For \ac{IoT} applications that are using the edge computing architecture, the popular
approach is to use \ac{OCI}-containers (e.g. Docker) to package and update
their applications.
However, a large scale study conducted by Zhao et al. showed that 80\% of the
70,000 most popular \ac{OCI}-container images on \textit{Dockerhub} are more than
190 \ac{MB} in size, and contain a high amount of duplicate files shared across the images
and layers, especially for the images which are larger than 190 \ac{MB} \cite{9242268}.
This has the serious implications that updating \ac{OCI}-container images with
the existing methodology on a device with limited bandwidth is also impractical
and challenging.

\section{Strategy}
To address the problem of reliably updating \ac{IoT} and \ac{IIoT}
devices with limited bandwidth, we need to investigate if using a different
update strategy in contrast to A/B partitioning can reduce the amount of data transmitted.
We have selected \textit{reUpNix} as the operating system to investigate, as its
update mechanism is specifically designed for \ac{IoT} and embedded devices with
limited bandwidth and storage \cite{gollenstede:23:lctes}. However, an integration
with a popular \ac{IoT} platform cloud services and \textit{reUpNix} has not
been done yet.

As a cloud platform for \ac{IoT} we have chosen Microsoft's \textit{Azure IoT}
as it is one of the most popular cloud platforms for \ac{IoT} and \ac{IIoT} applications.
\textit{Azure IoT Edge} in particular extends \textit{Azure IoT} to run
\ac{OCI}-containers on \ac{IoT} devices, which is a popular approach for edge
computing.

We need to find a strategy to update the \ac{OS} of \ac{IoT} device running
\textit{Azure IoT Edge} using \textit{Azure IoT Device Update Service} with
\textit{reUpNix}'s update mechanism. The strategy needs to reduce the transmitted data
for an update significantly in comparison to using a full \ac{OS} image in an
A/B partition schema. Additionally, we need to verify that using a different
\ac{OS} than the one recommended by Microsoft does not significantly impact
the reliability and performance of the \ac{IoT} device.


Not only the \ac{OS} needs to be updated, but also the \ac{IoT} applications
running as \ac{OCI}-containers on the device. To achieve this, we need to find a strategy
using \textit{reUpNix}'s \ac{OS} update mechanism to reduce the transmitted data
size in comparison to using \code{docker pull}. The proposed strategy must
reliably save bandwidth for the popular \ac{OCI}-container images from
\textit{Dockerhub}, while also not significantly increasing the disk storage
requirements on the device and integrating well with the existing container
runtime and \textit{Azure IoT Edge}.

