\chapter{Introduction}
\label{sec:introduction}


% \section{Context}
Since its first mention in 2006 \cite{Adelmann2006ToolkitFB} the \ac{IoT}
is a growing industry sector and active research field. A 2018 study conducted
by Dachyar et al., shows that \ac{IoT} has been continuously growing, is expected
to grow further, and will rise in demand for research and commercial application
because of its function to connect a variety of devices. The primary industries
where \ac{IoT} devices and applications are used today are manufacturing,
agriculture, public service, electronics, health, energy, and mining.
While most research is focused on topics such as security, protocols, energy and
power consumption \cite{dachyar2019knowledge}.

\ac{IoT} devices that are deployed in the mining and energy industry are often
installed in locations with limited, unreliable, or expensive network
connectivity, such as off-shore oil rigs or underground mining sites.
Some other industries have similar challenging installation sites, such as
satellites, cargo ships, and remote weather stations.
The limited bandwidth available in these remote locations poses a significant
obstacle for transmitting large amounts of data reliably and efficiently,
requiring innovative solutions for data optimization.

To address these challenges, there are already several approaches to compress
the transmitted data without a loss of information, which have shown to reduce
the data transmitted, but also require more computing power on the \ac{IoT}
devices \cite{9243457}. An additional technique to reduce the amount of data,
is to use a more efficient serialization format, such as Google's \textit{ProtoBuf}
which has shown to reduce the size of data transmitted drastically in comparison
to \textit{JSON} or \textit{XML} \cite{7765670}. Lastly, when switching to a
edge computing approach, the data can be preprocessed and aggregated before
sending it to the cloud, which reduces the amount of data transmitted.


\section{Problem}

As applications often gain functionality and complexity with each release, the
size of software may also increase. \ac{IoT} devices operating in environments
with limited bandwidth will face increasing challenges. One case where this is
particularly evident is the \textit{Linux} kernel used by many \ac{IoT} and embedded
devices, which expands in size with every new release \cite{linux-kernel-report}.

Research has shown that updating security issues in \ac{IoT} systems as quickly
and as frequently as possible significantly enhances the security of these systems.
This is especially important for \ac{IIoT} devices, which have an extended attack
surface and where an outage can impose physical and financial damage \cite{s20247160}.
For \ac{IoT} devices with limited connectivity, crucial security updates can be
significantly delayed or even impossible to apply, as the update size may exceed
the available bandwidth or disk storage. But not just the security of \ac{IoT}
devices is improved by frequent updating. Anand et al. showed in 2018 that the
performance and reliability of \ac{IoT} applications can be significantly improved
by updating frequently \cite{Anand2018}.

\bigskip\noindent Note: A B Updates are even worse

\bigskip\noindent Note: Docker Updates also not great

\bigskip\noindent Average Docker Container size?


\section{Strategy}
\begin{tcolorbox}[title=TODO]
    \begin{itemize}
    \item Creating NixOS derivations for Azure IoT Edge
    \item Integration the previously created modules into the minified ReUpNix
    \item Measure the size of the image
    \item Measure the time it takes to reboot the device
    \item Measure the size of the update
    \end{itemize}
\end{tcolorbox}
