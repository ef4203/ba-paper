\addcontentsline{toc}{chapter}{Abstract}
\begin{otherlanguage*}{american}

The growth of the \ac{IoT} in industries like production, manufacturing, and
agriculture has led to a wide range of innovations and new products. Currently
all major cloud providers such as Microsoft, Google, Amazon's AWS and IBM offer
\ac{IoT} services and solutions to their customers. However, many \ac{IIoT} devices,
which are installed in remote locations, such as offshore oil rigs or airplanes,
have limited, unstable, or costly network connections. Techniques to
minimize the \ac{IIoT} software's network traffic both in message size
(such as \textit{ProtoBuf} or other forms of compression), and in message amount
(such as local processing with edge computing) have already been established.
However, the image size, update, and consistency of the underlying operating
system is still a challenge. With Microsoft's \textit{Azure IoT Device
Update} service for example, the entire image for the operating system is sent
to the device and then written to a secondary partition. This results in hundreds
of megabytes being transmitted.

In this study, we investigated if the \textit{NixOS}-based \textit{ReUpNix}
can be used as a suitable host operating system for Microsoft's commercial
\ac{IoT} platform \textit{Azure IoT Edge}, since its image size is significantly
smaller than Microsoft's recommended Linux images.
 For this we identified the
minimum set of components and features that have to be added to heavily minified
\textit{ReUpNix} in order to have the required functionality for the
Docker-container-based \textit{Azure IoT Edge}, while remaining at a small
footprint in disk size.

With \textit{ReUpNix} as a host operating system for \textit{Azure IoT Edge}, we
observed 43\% smaller image size due to the minification and 90\% smaller size of
system updates due to the fact that only differences are sent, compared to
Microsoft's recommended \textit{Ubuntu 22.04}
image. Further, we showed that it is possible to pre-install Docker container
layers on the device, such that the amount of data that needs to be transmitted
in the case of a container update with \textit{Azure IoT Edge} is heavily reduced.
However, \textit{NixOS}-based systems do not receive official support from
Microsoft and product managers need to decide if the benefits outweigh the loss
of Microsoft's support.

\end{otherlanguage*}


\chapter*{Kurzfassung}
\addcontentsline{toc}{chapter}{Kurzfassung}
\begin{otherlanguage*}{ngerman}
Das Wachstum des Internet of Things (IoT) in Branchen wie Produktion, Fertigung
und Landwirtschaft hat zu einer Vielzahl von Innovationen und neuen Produkten geführt.
Derzeit bieten alle großen Cloud-Anbieter wie Microsoft, Google, Amazons AWS und IBM
ihren Kunden IoT-Dienste und Lösungen an.
Allerdings gibt es einige industrielle IoT-Geräte (IIoT), welche
 an abgelegenen Orten installiert sind, wie z. B. auf Bohrinseln oder in Flugzeugen,
die begrenzte, instabile oder teure Netzwerkverbindungen haben.
Methodiken zur Minimierung des Netzwerkverkehrs von IIoT-Software sowohl in der
Nachrichtengröße (wie ProtoBuf oder  Kompression) als auch in der Anzahl der Nachrichten
(z. B. lokale Verarbeitung mit Edge Computing) wurden bereits etabliert.
Allerdings bleiben die Größe, das Update und die Konsistenz des zugrunde liegenden
Betriebssystems eine Herausforderung. Mit Microsofts Azure IoT Device Update Service
wird beispielsweise das gesamte Image für das Betriebssystem an das Gerät gesendet
und dort auf eine sekundäre Partition geschrieben. Dies führt zu hunderten von Megabytes
die übertragen werden.

In dieser Studie untersuchen wir, ob das NixOS-basierte ReUpNix als geeignetes
Betriebssystem für Microsofts kommerzielle IoT-Plattform
IoT Edge genutzt werden kann, da das Betriebssystem signifikant kleiner ist als Microsofts empfohlenes
Betriebssystem Ubuntu 22.04. Dafür haben wir die minimalen Komponenten und Funktionen,
die zum stark minimierten ReUpNix hinzugefügt werden müssen, identifiziert, um die
Docker-Container-basierte Azure IoT Edge Runtime zu unterstützen, während wir
weiterhin einen kleinen Fußabdruck auf der Festplatte beibehalten.

Mit ReUpNix als Host-Betriebssystem für Azure IoT Edge haben wir eine 43\%
kleinere Imagegröße aufgrund der Minimierung und 90\% kleinere Größe von
Updates, die gesendet werden müssen, da nur die Unterschiede in den Dateien gesendet
werden, im Vergleich zu Microsofts empfohlenem Ubuntu 22.04 Image. Weiterhin haben
wir gezeigt, dass es möglich ist, Docker-Container-Layer auf dem Gerät vorzuinstallieren,
so dass die Menge an Daten, die im Falle eines Container-Updates mit Azure IoT Edge
übertragen werden muss, stark reduziert wird. Allerdings erhalten NixOS-basierte
Betriebssysteme keine offizielle Gewährleistung von Microsoft und Produktmanager
müssen entscheiden, ob die Vorteile den Verlust der Gewährleistung von Microsoft überwiegen.

\end{otherlanguage*}
