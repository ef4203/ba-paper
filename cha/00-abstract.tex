\chapter*{Abstract}
\addcontentsline{toc}{chapter}{Abstract}
\begin{otherlanguage*}{american}

The growth of the \ac{IoT} in industries like production, manufacturing, and agriculture has led to a
wide range of innovations and new products. Currently all major cloud providers such as Microsoft, Google,
Amazon's AWS and IBM offer \ac{IoT} services and solutions to their customers.
However many \ac{IIoT} devices that are installed in remote locations, such as offshore oil
refineries or airplanes, have limited, unstable, or costly network connections. While techniques to minimize the \ac{IIoT} software’s
network traffic both in message size (such as \textit{Protobuf} or other forms of compression) and in
message amount (such as local processing with edge computing) have already been established. However the image size,
update, and consistency of the underlying operating system is still a challenge. With Microsoft's \textit{Azure IoT Device
Update} service for example, the entire image for the operating system is sent to the device and then
written to a secondary partition. This results in hundreds of megabytes being transmitted.

In this field study, we investigate if the \textit{NixOS}-based \textit{ReUpNix} can be used as a suitable host operating
system for Microsoft’s commercial \ac{IoT} platform \textit{Azure IoT Edge} since its image size is significantly smaller
than Microsoft's recommended \textit{Ubuntu 22.04} image. Further, we investigate the conceptual advantages for \ac{IoT}
devices that \textit{NixOS}-based systems provide over traditional \textit{Linux} operating systems, when altering configuration.
For this, we identified the minimum set
of components and features that have to be added to heavily minified \textit{ReUpNix} in order to have the required
functionality
for the OCI-container-based \textit{Azure IoT Edge}, while remaining at a small footprint in disk size and memory
usage.

\textbf{[Expected result]}
With \textit{ReUpNix} we observed X\% lower image size, Y\% faster application update times, and
an equal amount of telemetry messages lost compared to Microsoft’s recommended \textit{Ubuntu 22.04}
image. However \textit{NixOS}-based systems do not receive official support from Microsoft and product
managers need to
decide if the benefits outweigh the loss of Microsoft's support.


\begin{tcolorbox}[title=TODO]
about 1/2 page:   \\
(1) Motivation (Why do we care?)   \\
(2) Problem statement (What problem are we trying to solve?)   \\
(3) Approach (How did we go about it)   \\
(4) Results (What's the answer?)   \\
(5) Conclusion (What are the implications of the answer?)\\
\end{tcolorbox}


\end{otherlanguage*}


%\chapter*{Kurzfassung}
%\addcontentsline{toc}{chapter}{Kurzfassung}
%\begin{otherlanguage*}{ngerman}

%Gleicher Text in Deutsch

% \end{otherlanguage*}
