\chapter{Architecture}
\begin{tcolorbox}[title=TODO]
Developed architecture / system design / implementation: 1/3

\begin{itemize}
\item start with a theoretical approach
\item describe the developed system/algorithm/method from a high-level point of view
\item go ahead in presenting your developments in more detail
\end{itemize}
\end{tcolorbox}


\section{Azure IoT Edge Nix Package}

\section{Experiments}

\subsection{Base Image Size}

\subsection{Update Image size}
\label{sec:update-image}
When updating the operating system with an A/B failover model, the entire
image needs to be downloaded and written to a secondary partition. In this
\ac{OTA} scenario the size of the operating system image is critical.
For this experiment two sizes are relevant, when comparing operating systems.
Firstly, the actual size of the image that needs to be written to the partition.
Secondly, the size that the entire system takes up on the disk after booting.

\subsection{Boot times}
Customers of certain industries require high availability and reliability for
their \ac{IIoT} devices and applications. These availability requirements
are commonly legally and formally negotiated in \ac{SLA}s between
the customer and a service provider \cite{msdoc-slas}.

An important consideration for service providers to achieve high
availability is the boot time of the \ac{OS}. However for this experiment the
time until operability is considered, instead of the actual boot time of the
\ac{OS}. The time until operability is more relevant for service providers
when defining \ac{SLO}.
For this experiment an outage of the service will be simulated by sending a
\textit{reboot} system call to the kernel with the \textit{reboot}
command line tool. It is important to note, that the "--force" option will
instruct the command line tool to not call the \textit{shutdown} system call,
which results in a faster shutdown of the system \cite{man-reboot}\cite{man-shutdown}.
Just before the system is rebooted, the current time will be printed.
\\

\begin{lstlisting}[caption=Command to print the current time and reboot]
date +%s && reboot -f
\end{lstlisting}
After the \ac{OS} rebooted and \textit{systemd} has started, the
\textit{Sample Application Telemetry Service} will send the current timestamp
to the \textit{Azure IoT Hub} where it can be retrieved. When comparing
the first timestamp from the shell command with the second timestamp from
the \textit{Azure IoT Hub}, the time until operability can be calculated.
This experiment must be repeated until the standard deviation is low enough
to have a confident mean value for each \ac{OS}.

\subsection{Container Updates}



\section{Build Host}
During the experiments conducted in this thesis, it was
required to compile, package, and build various software.
Also, the operating system variants required a
computationally expensive creation of an installable image.
For building and testing the various operating systems and
software packages the following hardware was used:
\begin{itemize}
    \item Processor: AMD Ryzen 7 5700X,
    \item Memory: 32 GB,
    \item Storage: 512 GB SSD,
\end{itemize}

Further, the following software versions were used on the build host:
\begin{itemize}
    \item Linux 6.5.6,
    \item Git 2.42.0,
    \item Nix 2.18.1,
    \item GCC 13.2.1 20230801,
    \item Make 4.4.1.
\end{itemize}
