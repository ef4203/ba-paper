\chapter{Conclusion}

\section{Summary}
\begin{tcolorbox}[title=TODO]
Summarize again what the paper did, but now emphasize more the results, and comparisons
\end{tcolorbox}

\section{Conclusion}
This section will highlight the main findings of the thesis to provide a
conclusion, so that \ac{IIoT} application developer and product managers can
make a careful consideration when choosing an \ac{OS} for their \ac{IIoT} devices.

\subsection{Image And Update Size}
It was shown that reUpNix has the the potential to drastically reduce the transferred
data when updating the operating system, which is important for \ac{IIoT} devices
have a limited Internet connection.

\subsection{Updating Docker Containers}

\begin{tcolorbox}[title=TODO]
    Highlight how updating Docker containers with reUpNix is more efficient than
    updating them with \code{docker pull}.
\end{tcolorbox}

\subsection{Limited Microsoft Support}
Microsoft does not officially support NixOS or reUpNix, which may be a decisive factor
for product managers choosing an \ac{OS} for their \ac{IIoT} devices. Microsoft
only provides so called "Tier 1" support for Ubuntu, Debian, Red Hat Enterprise
Linux and Windows, which means that Microsoft is actively testing and validating
new releases. If a developer is using an \ac{OS} that is not officially supported
and encounters a problem, they need to reproduce the issue with one of Microsoft's
"Tier 1" supported \ac{OS} before they can open a support ticket with Microsoft
\cite{msdoc-supportetplatforms}.

\section{Future Work}

\subsection{Podman}
Podman is a container runtime developed by Red Hat and a direct competitor to Docker.
In the future, it would be interesting to investigate the possibility of using
Podman instead of Docker as the container runtime for Azure IoT Edge. Podman is
not officially supported by Microsoft, but it can potentially used as a
replacement for Docker, since it can provide a \ac{UDS} for the IoT Edge runtime
to communicate with the container runtime \cite{book:3556946,msdoc-supportetplatforms}.

\subsection{Implementing A Update Agent Handler}
\begin{tcolorbox}[title=TODO]
Microsoft allows to implement their own update agent handler, which could be
used to implement a custom update agent handler that is compatible with reUpNix.
\end{tcolorbox}

\subsection{Further minification}
\begin{tcolorbox}[title=TODO]
Elaborate that this thesis has NOT shown the absolute smallest possible system
to run Azure IoT Edge on, and that there's still room for further minification.
\begin{itemize}
    \item Minify Microsoft's software e.g. by compiling with \code{-w} and \code{-s} flags
    \item Or using software like \textit{upx} to compress binaries
    \item Minify Podman and/or Docker runtime
\end{itemize}
\end{tcolorbox}

\subsection{Code optimization}
As with most software projects, the written source code for this thesis could
almost always be optimized to be written to run faster, use less resources, be
more concise and be more readable. As the knowledge about the system and the
Nix language grows, and the helper functions added to Nix increase, the
written configuration files can be optimized. At the time of the research, the
Nix build process takes a lot of time and resources, and an optimization could
lead to an increase in productivity and less time spend waiting.
