\section{Summary}
In this thesis the possibility of using \textit{reUpNix} as the host \ac{OS} for
running the popular \textit{Azure IoT} platform on \ac{IIoT} devices has been investigated.
The focus of this study has been on the update mechanism of \textit{reUpNix} and how it can be used
to save bandwidth when updating the \ac{OS} and \ac{OCI}-container images on the device,
since some \ac{IoT} devices have limited bandwidth because they are installed in
remote locations and/or have limited storage. It could be shown that \textit{Azure IoT Edge}
can be packaged and run on the \textit{Nix}-based \textit{reUpNix} and
\textit{reUpNix}'s \textit{Nix store} does not introduce any significant overhead
or delay when recovering from a reboot. When comparing \textit{reUpNix} to
other \ac{OS} that are able to run \textit{Azure IoT Edge}, it has been observed that \textit{reUpNix}
had a significantly smaller image size than the other \ac{OS}. In particular,
\textit{reUpNix} has a 43\% smaller image size than for example Microsoft's
recommended \textit{Ubuntu 22.04}. It has also been shown that \textit{reUpNix}'s update
methodology is significantly more efficient than the commonly used A/B partition
schema. In the tests, \textit{reUpNix} was able to update the \ac{OS} and save
90.3\% of the transferred data compared to a A/B partition schema with \textit{Ubuntu 22.04}.


\section{Conclusion}
It was shown that reUpNix has the the potential to drastically reduce the transferred
data when updating the operating system, which is important for \ac{IIoT} devices
that have a limited Internet connection. \textit{reUpNix} provides a viable
alternative for \ac{IoT} devices running \textit{Azure IoT Edge}
to the commonly used A/B partition schema, which is used by most platforms,
including the \textit{Azure Device Update Service}.

However, Microsoft does not officially support NixOS or reUpNix, which may be a decisive factor
for product managers choosing an \ac{OS} for their \ac{IIoT} devices. Microsoft
only provides so called "Tier 1" support for Ubuntu, Debian, Red Hat Enterprise
Linux and Windows, which means that Microsoft is actively testing and validating
new releases. If a developer is using an \ac{OS} that is not officially supported
and encounters a problem, they need to reproduce the issue with one of Microsoft's
"Tier 1" supported \ac{OS} before they can open a support ticket with Microsoft
\cite{msdoc-supportetplatforms}.

Lastly, this study showed that using
\textit{reUpNix}'s differential updates to update \ac{OCI}-container images
does not reduce the data transmitted drastically for all images, compared to using
\code{docker pull}. However, the \textit{Nix store} can be used to store the images
and it is possible to load the images from the \textit{Nix store} into \textit{Docker}.


\section{Future Work}
In order to store the \ac{OCI}-container images in the \textit{Nix store} and
handle the update of the images easier, a custom storage provider for docker images
that uses the \textit{Nix store} as the backend could be implemented.
This would eliminate some of the complexity of the current solution and reduce
the overhead required to load the container images.

Further, the possibility of implementing a custom update handler for
the \textit{Azure Device Update Agent} could be investigated, since Microsoft allows
the implementation of custom update handlers. This would allow the replacement of the
current script handler with a dedicated update handler that is compatible
with \textit{reUpNix}'s update mechanism and has lower overhead.

Lastly, it has not been shown that the system that has been built is the smallest possible
system to run \textit{Azure IoT Edge} on. There are still possibilities to explore, which
further minify the system, such as using \textit{Podmand} instead of \textit{Docker}
or minifying the \textit{Azure IoT Edge} runtime and the \textit{Azure IoT Edge} modules
or using software like \textit{upx} to compress binaries.
