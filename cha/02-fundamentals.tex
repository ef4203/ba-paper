\chapter{Fundamentals}
\label{sec:fundamentals}
This chapter introduces the major concepts, technologies,
and software relevant to this thesis. First, it will introduce the operating
systems \textit{Ubuntu 22.04}, \textit{Yocto Kirkstone Linux}, \textit{NixOS},
and \textit{ReUpNix}, which we will use for comparison in the later experiments.
Further, we will introduce the \textit{Azure IoT Edge} platform by Microsoft,
with all the components that were used.
Finally, the last sections will give an overview about the related work.

\section{Technology}
\subsection{Ubuntu 22.04}
\textit{Ubuntu 22.04} is the latest long term support release of the
\textit{Linux}-based operating system \textit{Ubuntu} developed by Canonical.
It will receive updates and support for the next 5 years, until April
2027\cite{ubuntu-releasenote}. Additionally it is also one of Microsoft's
recommended operating systems to run containerized applications with
\textit{Azure IoT Edge} on Linux\cite{msdoc-supportetplatforms}.

\subsection{Yocto Kirkstone Linux}
\begin{tcolorbox}[title=TODO]
    \begin{itemize}
        \item Describe Yocto
    \end{itemize}
\end{tcolorbox}

\subsection{NixOS}
\textit{NixOS} is a \textit{Linux}-based operating system that uses the purely
functional package manager \textit{Nix}. It is currently maintained as
an open source project by the non-profit organization \textit{NixOS Foundation}
and the \textit{NixOS Community}. The project started in 2003 by Eelco Dolstra,
who is the current president of the \textit{NixOS Foundation}, as
a research project at the Utrecht University\cite{dolstra2003} and was introduced again in his 2008
paper "NixOS: A Purely Functional Linux Distribution".

\begin{tcolorbox}[title=TODO]
Describe the following concepts:
\begin{itemize}
    \item Declarative Configuration
    \item Immutable System
    \item Functional Package Management
    \item Reproducible Builds
    \item Rollback and Atomic Upgrades
\end{itemize}

    Cite information from this paper: \cite{1411255}
\end{tcolorbox}

\subsection{ReUpNix}

\begin{tcolorbox}[title=TODO]
    \begin{itemize}
        \item Highlight differences to NixOS
        \item Cite information from this paper: \cite{gollenstede:23:lctes}
    \end{itemize}
\end{tcolorbox}


\subsection{Azure IoT}
\textit{Azure IoT} is a cloud platform provided by Microsoft that
empowers organizations to build, deploy, and manage \ac{IoT} solutions at scale.
\textit{Azure IoT} offers a suite of services and tools designed to connect,
monitor, and control a diverse array of devices, sensors, and equipment. The
platform enables integration of \ac{IoT} devices with Microsoft's other cloud
services, facilitating the collection and analysis of data for actionable insights.
This ecosystem supports a wide range of industries, from manufacturing and
healthcare to transportation and smart cities\cite{msdoc-aziot}.

While the "big three" cloud service providers, namely Microsoft, Amazon and
Google, all offer \ac{IoT} platforms and services, a conducted comparison finds
that Microsoft's \textit{Azure IoT} provides the most tools for businesses,
especially data visualization tools. However, Amazon's \textit{AWS} \ac{IoT}
offerings have the largest market share\cite{9116254}.

\subsubsection{Azure IoT Identity Service}
\begin{tcolorbox}[title=TODO]
    \begin{itemize}
        \item Describe Azure IoT Identity Service
    \end{itemize}
\end{tcolorbox}

\subsubsection{Azure IoT Edge}
The \textit{Azure IoT Edge} platform is a collection of software products and
services developed and maintained by Microsoft that extends the capabilities
of \textit{Azure IoT} to the edge of the network. This platform facilitates the
deployment and management of \ac{IoT} devices, enabling them to run \ac{AI},
machine learning, and analytics locally. Azure IoT Edge allows organizations to
process and analyze data on-site, near the data source, reducing latency and
optimizing bandwidth usage.
Unlike traditional cloud computing, where data is sent to a centralized server
for analysis, edge computing occurs on or near the device or "edge" of the
network. This approach is particularly valuable in scenarios where quick
response times are critical or where excessive data transfer to the cloud
is not possible\cite{msdoc-aziotedge}.

All software, as well as \ac{SDK}s and
libraries which are installed on the device are open source, however the
cloud services offered by Microsoft are almost exclusively closed source.
Development and maintainance of \textit{Azure IoT Edge} is publicly visible on
\textit{GitHub}. The biggest contributions
to the public repository are done by Microsoft employees Varun Puranik,
Philip Lin and Damon Barry, but it also features bug-fixes and contributions
from the community and non-Microsoft employees
\footnote{Full overview of all contributions: https://github.com/Azure/iotedge/graphs/contributors}.

\subsubsection{Azure IoT Edge Modules}
In Microsoft's terminology, containerized workloads which will be run on the \ac{IIoT}
devices are called \textit{IoT Edge Modules}. These modules "can be configured to
communicate with each other, creating a pipeline of data processing." However developers are
not limited to only Microsoft's services, instead, they can deploy their own
applications as long as they are running in \ac{OCI}-compliant \textit{Linux} containers.
Microsoft calls this approach "bring your own code" and offers are variety of
\ac{SDK}s for popular programming languages such as .NET, Java, C and Python\cite{msdoc-supportetplatforms}.

\subsubsection{Azure IoT Edge Agent}
\begin{tcolorbox}[title=TODO]
    \begin{itemize}
        \item Describe the Edge Agent
    \end{itemize}
\end{tcolorbox}

\subsubsection{Azure IoT Hub}
\begin{tcolorbox}[title=TODO]
    \begin{itemize}
        \item Describe the Iot Hub
    \end{itemize}
\end{tcolorbox}

\subsubsection{Azure IoT Device Update Service}
\begin{tcolorbox}[title=TODO]
    \begin{itemize}
        \item Describe ADU
    \end{itemize}
\end{tcolorbox}

\section{Related Work}
\begin{tcolorbox}[title=TODO]
    \begin{itemize}
        \item reUpNix
        \item NixOS
        \item Azure IoT Edge
    \end{itemize}
\end{tcolorbox}
