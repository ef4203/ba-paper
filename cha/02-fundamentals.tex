\chapter{Fundamentals}
\label{sec:fundamentals}
This chapter introduces the major concepts, technologies,
and software relevant to this thesis. First, it will
introduce the hardware and software used for the host system
on which the builds and tests have been conducted.
Secondly, it introduces the \textit{Azure IoT Edge} platform
by Microsoft, with all the components that were used.

\section{Build Host}
During the experiments conducted in this thesis, it was
required to compile, package, and build various software.
Also, the operating system variants required a
computationally expensive creation of an installable image.
For building and testing the various operating systems and
software packages the following hardware was used:
\begin{itemize}
    \item Processor: AMD Ryzen 7 5700X,
    \item Memory: 32 GB,
    \item Storage: 512 GB SSD,
\end{itemize}

Further, the following software versions were used on the build host:
\begin{itemize}
    \item Linux 6.5.6,
    \item Git 2.42.0,
    \item Nix 2.18.1,
    \item GCC 13.2.1 20230801,
    \item Make 4.4.1.
\end{itemize}



\section{NixOS}

\cite{1411255}
\section{ReUpNix}

\section{Azure IoT Edge}
Microsoft's \textit{Azure IoT Edge} platform is a collection of software products and services developed and
maintained by Microsoft to deploy, run, and manage containerized applications directly on
\ac{IoT} devices. It follows the Edge Computing paradigm,
which allows data processing and analysis to occur locally on the devices,
instead of sending all the collected real-time data to the cloud. Further advantages of this Edge Computing
architecture include the full utilization of the computing power on the device, low latency,
privacy, and temporary offline scenarios \cite{msdoc-aziotedge}. While all major cloud service providers
offer \ac{IoT} platforms and services, a comparison conducted finds that Microsoft's \textit{Azure IoT Edge} provides the most tools,
especially data visualization tools. However,
Amazon's \textit{AWS} \ac{IoT} offerings have the largest market
share \cite{9116254}.

\subsection{Azure IoT Edge Modules}
In Microsoft's terminology, containerized workloads which will be run on the \ac{IIoT}
devices are called \textit{IoT Edge Modules}. These modules "can be configured to
communicate with each other, creating a pipeline of data processing." Developers are
not limited to only Microsoft's services, instead, they can deploy their own
applications as long as they are running in \ac{OCI}-compliant \textit{Linux} containers.
Microsoft calls this approach \textit{Bring your own code} and offers are variety of
\ac{SDK}s for popular programming languages such as .NET, Java, C and Python \cite{msdoc-supportetplatforms}.

\subsection{Azure IoT Edge Agent}
\subsection{Azure IoT Hub}


\section{Ubuntu 22.04}
\textit{Ubuntu 22.04} is the latest long term support release of the \textit{Linux}-based operating system \textit{Ubuntu}. Which will receive updates and support
for the next 5 years, until April 2027 \cite{ubuntu-releasenote}. It is also one of
Microsoft's recommended operating systems to run containerized applications with
\textit{Azure IoT Edge} on Linux \cite{msdoc-supportetplatforms}.

\section{Test Setup}
\subsection{Hardware}
\subsection{Operating System}
\subsection{Containerized Application}
\section{Methodology}
\section{Related Work}
