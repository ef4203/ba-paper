This chapter introduces the major concepts, technologies,
and software relevant to this thesis. First, it will introduce the operating
systems \textit{Ubuntu 22.04}, \textit{Yocto Kirkstone Linux}, \textit{NixOS},
and \textit{ReUpNix}, which we will use for comparison in the later experiments.
Further, we will introduce the \textit{Azure IoT Edge} platform by Microsoft,
with all the components that were used.
Finally, the last sections will give an overview about the related work.

\section{Technology}
\subsection{Ubuntu 22.04}
\textit{Ubuntu 22.04} is the latest long term support release of the
\textit{Linux}-based operating system \textit{Ubuntu} developed by Canonical.
It will receive updates and support for the next 5 years, until April
2027 \cite{ubuntu-releasenote}. Additionally it is also one of Microsoft's
recommended operating systems to run containerized applications with
\textit{Azure IoT Edge} on Linux \cite{msdoc-supportetplatforms}.

\subsection{Yocto Kirkstone Linux}
The \textit{Yocto Project} is an open-source collaboration project that provides tools
and resources for building custom Linux-based systems for embedded devices.
\textit{Yocto Kirkstone Linux} is a recent long-term support release of the
\textit{Yocto Project}, which will receive support until April 2026 \cite{yocto-releases}.

\subsection{NixOS}
\textit{NixOS} is a \textit{Linux}-based operating system that uses the purely
functional package manager \textit{Nix}. It is currently maintained as
an open source project by the non-profit organization \textit{NixOS Foundation}
and the \textit{NixOS Community}. The project started in 2003 by Eelco Dolstra,
who is the current president of the \textit{NixOS Foundation}, as
a research project at the Utrecht University \cite{dolstra2003} and was introduced
as \textit{NixOS} in his 2008 paper "NixOS: A Purely Functional Linux Distribution".

A \textit{NixOS} system is configured with a set of configuration files, which
are evaluated by the \textit{Nix} package manager to produce a complete system.
The configuration files are written in the functional \textit{Nix} programming language,
which has the advantage of being declarative and reproducible. Further, \textit{Nix}
utilizes cryptographic hashes to verify the integrity of the inputs and outputs
for any given build. This means that the same configuration inputs produce the
same system. This approach aims to improve the correctness and reliability of
system deployments over traditional system build tools \cite{dolstra2006}.

A key concept of \textit{NixOS} is the \textit{Nix Store}, where all packages,
files, configurations, and other artifacts that the system requires are stored.
The \textit{Nix Store} is a immutable directory tree, where every \textit{Nix Store}
object is uniquely identified by its content hash and additionally a
human-readable name and version. In contrast to other operating systems where the package
manager is only used to manage the installation and removal of packages, in \textit{NixOS}
it is also used to manage the system configuration (typically stored in the
\textit{/etc} directory) and other static parts of the
system \cite{1411255}.

Lastly, \textit{NixOS} has the possibility to perform a rollback of the system to a
previous configuration. \textit{NixOS} will keep the previous configuration
in a list of generations for this purpose \cite{1411255}, allowing users to easily
revert to a known stable state in case of errors or undesired changes. This
feature provides an efficient way to manage system updates and changes over time.

\subsection{ReUpNix}
\textit{reUpNix} is a research project at the Hamburg University of Technology,
which aims to improve the shortcomings of \textit{NixOS} specifically for
embedded and \ac{IoT} devices. It was introduced a the International Conference on
\ac{LCTES} in 2023 and is primarily maintained by Niklas Gollenstede.

Since \textit{NixOS} is a general-purpose operating system, it is not optimized
for embedded and \ac{IoT} devices. As a result of this, the base image size of
\textit{NixOS} is relatively large, since it aims to support a wide range of
software and hardware for different use cases. \textit{reUpNix} aims to address
this issue by providing a minification process, which removes unnecessary
files, configurations and applications from the base image \cite{gollenstede:23:lctes}.

Additionally, it addresses the problem of the update size, which is an important
factor for \ac{IoT} devices, since they might have limited bandwidth and
potentially unreliable connections. The regular update mechanism for \textit{Nix}
will compute which dependencies are required and are not already present in the \textit{Nix Store}.
Once the dependencies are computed, the required packages are downloaded and
placed in the \textit{Nix Store}, regardless of how many files where changed inside
the package. In a desktop or server environment, this
update process is typically not an issue, since these devices are usually
connected via high speed internet connections. In contrast, an \ac{IoT} device
would waste a lot of bandwidth and time to download packages in which only a
small set of files have changed. \textit{reUpNix} addresses this issue by
replacing the regular \textit{Nix} update mechanism with a custom update mechanism
which only downloads the difference and applies them to the \textit{Nix Store}
\cite{gollenstede:23:lctes}.

Finally, \textit{reUpNix} provides a way to declaratively define and update
the configuration for the bootloader as well as disk partitioning. This wasn't
covered by \textit{NixOS} before, and is a crucial feature for embedded and
\ac{IoT} devices \cite{gollenstede:23:lctes}.

\subsection{Azure IoT}
\textit{Azure IoT} is a cloud platform provided by Microsoft that
empowers organizations to build, deploy, and manage \ac{IoT} solutions at scale.
\textit{Azure IoT} offers a suite of services and tools designed to connect,
monitor, and control a diverse array of devices, sensors, and equipment. The
platform enables integration of \ac{IoT} devices with Microsoft's other cloud
services, facilitating the collection and analysis of data for actionable insights.
This ecosystem supports a wide range of industries, from manufacturing and
healthcare to transportation and smart cities \cite{msdoc-aziot}.

While the "big three" cloud service providers, namely Microsoft, Amazon and
Google, all offer \ac{IoT} platforms and services, a conducted comparison finds
that Microsoft's \textit{Azure IoT} provides the most tools for businesses,
especially data visualization tools. However, Amazon's \textit{AWS} \ac{IoT}
offerings have the largest market share \cite{9116254}.

\subsubsection{Azure IoT Identity Service}
The \textit{Azure IoT Identity Service} is a set of system services developed
by Microsoft that \ac{IoT} developers can leverage to build applications which
communicate with the \textit{Azure IoT} cloud platform. They provide abstractions
for provisioning, managing, and authenticating devices, as well as managing
certificates and secrets on the device \cite{aiiot}. The services are implemented
in the Rust programming language and run under Linux-based system as \textit{systemd}
services.


\subsubsection{Azure IoT Edge}
The \textit{Azure IoT Edge} platform is a collection of software products and
services developed and maintained by Microsoft that extends the capabilities
of \textit{Azure IoT} to the edge of the network. This platform facilitates the
deployment and management of \ac{IoT} devices, enabling them to run \ac{AI},
machine learning, and analytics locally. Azure IoT Edge allows organizations to
process and analyze data on-site, near the data source, reducing latency and
optimizing bandwidth usage.
Unlike traditional cloud computing, where data is sent to a centralized server
for analysis, edge computing occurs on or near the device or "edge" of the
network. This approach is particularly valuable in scenarios where quick
response times are critical or where excessive data transfer to the cloud
is not possible \cite{msdoc-aziotedge}.

All software, as well as \ac{SDK}s and
libraries which are installed on the device are open source, however the
cloud services offered by Microsoft are almost exclusively closed source.
Development and maintainance of \textit{Azure IoT Edge} is publicly visible on
\textit{GitHub}. The biggest contributions
to the public repository are done by Microsoft employees Varun Puranik,
Philip Lin and Damon Barry, but it also features bug-fixes and contributions
from the community and non-Microsoft employees
\footnote{Full overview of all contributions: https://github.com/Azure/iotedge/graphs/contributors}.

\subsubsection{Azure IoT Edge Modules}
In Microsoft's terminology, containerized workloads which will be run on the \ac{IIoT}
devices are called \textit{IoT Edge Modules}. These modules "can be configured to
communicate with each other, creating a pipeline of data processing." However developers are
not limited to only Microsoft's services, instead, they can deploy their own
applications as long as they are running in \ac{OCI}-compliant \textit{Linux} containers.
Microsoft calls this approach "bring your own code" and offers are variety of
\ac{SDK}s for popular programming languages such as .NET, Java, C and Python
\cite{msdoc-supportetplatforms}.

\subsubsection{Azure IoT Edge Agent}
The \textit{Azure IoT Edge Agent} is a crucial component of the \textit{Azure IoT Edge}
platform, responsible for managing the lifecycle of \textit{IoT Edge Modules} and
handling communication between these modules and the \textit{Azure IoT Hub}.
It executes the deployment, configuration, and monitoring of modules running on
edge devices, by communicating with the \ac{OCI}-container runtime,
ensuring they function according to defined deployment and settings.
The \textit{Azure IoT Edge Agent} also leverages the \textit{Azure IoT Identity Service}
for secure communication, handling authentication and encryption for data exchange
between the edge devices and the cloud \cite{msdoc-aziotedge-arch}.


\subsubsection{Azure IoT Hub}
The \textit{Azure IoT Hub} is a cloud service developed by Microsoft designed to
enable secure and scalable communication between a fleet of \ac{IoT} devices and
the cloud. It serves as a central message queue, enabling devices to send telemetry
data, and be managed remotely. With features such as device registration,
authentication, and monitoring, the \textit{Azure IoT Hub} provides a reliable and efficient
way to connect and manage, a diverse fleet of \ac{IoT} devices, empowering
organizations and developers to build and deploy \ac{IoT} solutions at scale \cite{msdoc-aziothub}.

\subsubsection{Azure IoT Device Update Service}
The \textit{Azure IoT Device Update Service} is a cloud service developed by Microsoft
for managing and automating the firmware and system updates of \ac{IoT} devices at scale.
It enables organizations to securely deploy, monitor, and schedule updates for
a large number of connected devices, ensuring they are always up-to-date with
the latest security patches, bug fixes, and feature enhancements. This service
simplifies the management of device updates, offering capabilities such as
rollback options, group-based deployments, and detailed reporting to track the
status of updates across diverse \ac{IoT} device fleets. In order to use the service,
the devices must be running the \textit{Azure IoT Edge} runtime, as well as the
\textit{Azure IoT Device Update Agent} a \textit{systemd} service which is
responsible for the communication with the \textit{Azure IoT Device Update Service}
\cite{msdoc-adu}.


\section{Related Work}
\begin{tcolorbox}[title=TODO]
???
\end{tcolorbox}
